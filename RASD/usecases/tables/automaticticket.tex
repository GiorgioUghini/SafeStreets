\begin{table}[!htbp]
	\hypertarget{tab:AutomaticTrafficTicket}{}
	\centering
	\begin{tabular}{lp{10cm}}
\bf\large Name&\bf\large Automatic Traffic Ticket generation\\
\hline
\hline
\bf Actors&User\\
\hline
\bf Entry conditions&The user is logged in and is in the main page.\\
\hline
\bf Flow of events&
\begin{itemize}
\itemsep0em 
\item The user clicks on "Report Violation" button and is redirected to the input form to create a report.

\item The user fills up the form with the violation information (type of violation, pictures of it, address, description, license plate position, date and time, etc...). Eventually some fields such as date and time will be auto-completed.

\item The user clicks on the "Verified Violation" checkbox because he is sure of it being a violation because it's evident or any other motivation.

\item The user clicks on the "Send Report" button after a quick check of all the field he typed.

\item The system shows a confirmation message to the user and redirects him to the main page.

\end{itemize}
\\
\hline
\bf Exit conditions&The new report is created into the SafeStreets system and passed to the Local Police service.\\
\hline
\bf Exceptions&
\setlist{nolistsep}
\begin{itemize}
	\itemsep0em 
	\item The information inserted is wrong (non-existent address, date and time in the future) or some information is missing: a corresponding error is displayed and the user is asked to modify the inserted information accordingly.
	\item The violation results in a duplicate or the license plate is not readable: the system put this report in a revision queue and block the process.
\end{itemize}
\\
\hline

\end{tabular}
\caption{Automatic Traffic Ticket generation Use Case table}
 \label{tab:AutomaticTrafficTicket}
\end{table}