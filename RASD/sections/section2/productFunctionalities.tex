This section provides an abstract of the main functions of the application. To be able to use any of the given functionalities, the user must first register and then login to the application by providing a valid email and a password.

\subsubsection[User data management]{[F0] User data management\hypertarget{sec:f0}{}}
SafeStreets users must sign up the first time they intend to access any application functionality and further usages of the app will require a login to access all its functionalities. Of course, there will be the possibility to recover the password if a user does not remember it.

\subsubsection[Notification of Violations]{[F1] Notification of Violations\hypertarget{sec:f1}{}}
\label{sec:notification_of_violations}
The base function of the application is the possibility to signal a traffic violation.

The user must send one or more picture(s) of the car in which both the violation and the license plate are clearly visible -
the licence plate will be automatically read by the software and prompted to the user.

The application will try to automatically get the user position using its GPS system, and will notify the user in case of failure so that it can enter it manually.

The users will then send the following information to the system:
\begin{itemize}
    \item The pictures selected by the user
    \item The position of the user
    \item The current date and time
    \item The type of violation (to be picked from a pre-defined list)
    \item An optional description inserted by the user
\end{itemize}
The information sent by the user will be stored on persistent storage on the server and the officers will be able to see it on their clients.

\subsubsection[Data Mining]{[F2] Data Mining\hypertarget{sec:f2}{}}
\label{sec:data_mining}
The system will allow both users and officers to extract statistics about violations in the various areas/streets of the cities in the system,
for example a user can find the areas in which most reports occurred in the last 3 months.

In particular, a user will be able to get information about:
\begin{itemize}
    \item The app usage: how many users are using the app and when
    \item The effectiveness of the initiative by looking at trends
    \item The most popular traffic violations in a city by category
    \item The most active user of the month: the one who reported most violations
\end{itemize}
\subsubsection[Request for interventions]{[F3] Request for interventions\hypertarget{sec:f3}{}}
\label{sec:request_for_interventions}
The system will get information from the local police systems about incidents, including the location, the licence plates of the cars involved, and the infractions committed.

By crossing the data about the incidents with the segnalations from its users, SafeStreets will be able to find unsafe areas and also to suppose a reason for it and make suggestions.
For example, if a road has many cyclists hit and many signals of cars parked on the bike lane, it can suggest to add a barrier between the parking lane and the road.
The correlations between the infractions found on the police system and the ones on the SafeStreet system, along with the possible solutions, will be done automatically and can be revised by hand by a officer.
An artificial intelligence can then help to calibrate when the system should launch a warning, training on the approval/rejection of the previous signals.
The officers responsible for handling these recommendations will see on their clients all the data about the signal,
including the number of incidents and signals, and can decide to discard it or approve it, thus keeping it in the system for future reference.
All the approved signals will be reachable by the officers, once they have been resolved they can be deleted from the list but will remain in an archive available for the AI.


\subsubsection[Automatic Tickets]{[F4] Automatic Tickets\hypertarget{sec:f4}{}}
\label{sec:automatic_tickets}
The user will be indirectly able to give tickets when he spots a violation. This works by adding to the functionality
\hyperref[sec:notification_of_violations]{"Notification of Violations"}
the possibility to confirm that he is sure that he is signalling an actual violation and wants to give a ticket.
In this case the system reads the licence plate and uses the police system to automatically issue a ticket.

The tickets are first checked by a computer vision mechanism that takes several parameters (position, pictures,
similar violations) to spot wrong violation and forward them to an officer for a double check. The officer can then
accept the ticket or delete it.

Security is a key aspect of this functionality. The system must be sure that the signal has not been modified.
To achieve this, all messages are encrypted with TLS and an hash of every picture is stored on a server which will not
allow any modifications to it, and will be used by the police systems to confirm the tickets.

\subsubsection[Statistics]{[F5] Statistics\hypertarget{sec:f5}{}}
The system will also be able to build statistics using the violation and ticket systems.
These data will represent the amount of tickets given in a certain street/area in a given period, the most egregious offenders, and the change in the number of tickets in a certain place during time.
\clearpage