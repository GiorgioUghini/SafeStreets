\setlength{\parskip}{1em}
The idea is to create an application to allow users to report parking violations without taking too much time from their daily life.
\\We want therefore to realize an extremely friendly user interface and a lightweight software in order to make SafeStreets affordable to as many people as possible and to be able to run by many devices.

Users will certainly be able to exploit the advanced functions of SafeStreets such as charts and analitycs. However, since those functions rely on data, the basic violations reporting function will be the core one.

Since a small downtime of SafeStreets is not going to cause damage to anyone, it will be tolerated. On the other hand, as our software is going to run some kind of OCR and AI recognition algorithm that will probably be expensive in terms of resources, it should be very dynamic to support different queries in a few seconds.
\\In addition, our software is going to process very specific data that could potentially lead someone to be fined, hence it should ensure that the chain of custody of a report is never broken and the images are never altered.

To upload a new picture on SafeStreets or to view charts about violations, a functional internet connection is required.

The officers appointed to check SafeStreets violations reports will use an external CMS developed by the municipality, that will access SafeStreets data through APIs, since a dedicated section for the authorities directly on the app could be exploited by malicious users.

Concerning the hardware, we intend to have a storage system which contains all the historical information about reports made by the users, most likely a database or a data warehouse. This system will allow both users and officers to query for aggregated and detailed information which require a huge amount of data to be processed.
\clearpage