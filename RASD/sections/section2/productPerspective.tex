The idea is to create an application to allow users to report parking violations without taking much time to their daily life. According to this intention, we would like to realize an extremely friendly user interface and a lightweight software in order to make SafeStreets affordable to many people as possible and runnable by many devices.\\
Users will certainly be able to exploit the advanced functions of SafeStreets such as charts and analitycs, but as those functions rely over data, the basic violations reporting function will be the core one.\\
\\
Since a small downtime of SafeStreets is not going to cause damage to anyone, it will be tolerated without much thoughts. On the other hand, as our software is going to run some kind of OCR and AI recognition algorithm that will probably be expensive in terms of resources, it should be very dynamic to support different queries in a few seconds.\\
In addition, our software is going to process very specific data that could potentially lead someone to be fined, hence it should ensure that the chain of custody is never broken and the images are never altered.\\
To upload a new picture on SafeStreets or to view charts about violations, it is obviously required an active and functional internet connection. But as said, as data are the core business of SafeStreets, there will be put in place a mechanism such that a user can insert all the information needed to report someone on his mobile application, then those information will be sent as soon as the internet connection is restored.\\
\\
Concerning the hardware, we intend to have a database which contains all the historical information about reports made by the people. This database will allow both users and officers to see both aggregated and detailed information that require an huge amount of data to be processed. Hence, the internal database engineering should take this into consideration.\\