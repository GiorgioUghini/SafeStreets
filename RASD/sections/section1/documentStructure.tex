This document is divided in four parts:
\begin{itemize}
	\item \textbf{Introduction}: a description about the goals of SafeStreets and the context in which it will be implemented is provided. A subsection dedicated to the understaing of some acronyms and definitions is also present. 
	
	\item \textbf{Overall Description}: gives an overall description of SafeStreets, focusing on the domain assumptions and the constraints of the application. This section also aims to provide a context to the whole project and to show its integration with the real world. It also shows the possible interactions between the world and the users of SafeStreets. 
	
	\item \textbf{Specific Requirements}: the software requirements, explained in a sufficiently detailed manner to design a system that satisfies them, and the testers to test said requirements are provided.
		
		A detailed description of the possible interactions that can occur between the world and the system is also present, followed by a series of simulations and previews about the interactions mentioned above.
	
	\item \textbf{Formal Analysis using Alloy}: the requirements are expressed through the Alloy model, which, being it is a declarative specification language, makes it possible to define the functions, the constraints and the interactions of SafeStreets.
\end{itemize}
In the last part of the document a short note about the softwares used and the effort spent in producing this RASD by its authors is shown.