This document is divided in four parts:
\begin{itemize}
	\item \textbf{Introduction}: a description about the goals of SafeStreets and the context in which it will be implemented is provided. Subsections dedicated to the understaing of some acronyms and definitions are also present. 
	
	\item \textbf{Overall Description}: gives an overall description of SafeStreets, focusing on the domain assumptions and the constraints of the application. This section also aims to provide a context to the whole project and to show its integration with the real world. It also shows the possible interactions between the world and the users of SafeStreets. 
	
	\item \textbf{Specific Requirements}: the software requirements, explained in a sufficiently detailed manner to design a system that satisfy them, and the testers to test said requirements are provided. It is also present a detailed description of the possible interactions that can occur between the world and the system, followed with a series of simulations and previews about the above mentioned interactions.
	
	\item \textbf{Formal Analysis using Alloy}: the requirements are expressed through the Alloy model, with which is possible, since it is a declarative specification language, to define the functions, the constraints and the interactions of SafeStreets.   
\end{itemize}
In the last part of the document a short note that summarize the effort spent in producing this RASD by its authors is shown.