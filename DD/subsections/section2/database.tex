Each microservice will have its own data storage system but not all of them will result to be a database. In particular, those microservices who will need object storage for training purposes (such as Computer Vision and Data Mining) will use a transparent-to-us data layer.\\
However, to make SafeStreets data persistent these database will be deployed:
\begin{itemize}
	\item \textbf{Violations DB}: This database will store all the data regarding a single violation except for images. As the data inside it are well structured this database will be a relational database.
	\item \textbf{Statistics DB}: As opposed to the above database, the Statistics DB will be deployed to store all internal and external data regarding the usage of SafeStreets. Indeed, the mechanism chosen for such DB is the data warehouse container.
	\item \textbf{Suggestion DB}: Every time the Data Mine microservice runs, it will produce possible intervention to some areas considered dangerous. Such advises that comes from the Data Mining algorithm will be stored in a highly unstructured database, formerly in a NoSQL document-oriented database.
\end{itemize}
Obliviously, all the used systems must obey to the laws regarding data protection (like GDPR), which means that (list not exhaustive):
\begin{itemize}
\item Sensible data such as passwords and personal information must be encrypted properly before being stored in the selected microservice.
\item Users must be granted access only upon provision of correct and valid credentials.
\end{itemize}

Below we include some graphs to better understand all our microservices