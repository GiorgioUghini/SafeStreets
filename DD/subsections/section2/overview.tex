This section of the document gives a detailed view of the physical and logical infrastructure of the system-to-be. 
\\It provides the different types of view over the system as well as the description of the main components and their interactions.
The logical division of the application consists of 3 layers: presentation, application and data.
Every layer, then, is made of several microservices.
\begin{enumerate}
	\item Subsection \ref{presentationlayer} \textbf{Presentation layer}: The Presentation Layer is the space where interactions between humans and machines occur. The goal of this interaction is to allow effective operation and control of the machine from the human end, whilst the machine simultaneously feeds back information to the server.
	\item Subsection \ref{applicationlayer} \textbf{Application layer}: The Application Layer is the part of the program that encodes the real-world business rules that determine how data can be created, stored, and changed.
	\item Subsection \ref{datalayer} \textbf{Data layer}: The Data Layer is physically where the data are stored. On this layer happen all the retrieve and update operations on the data. 
\end{enumerate}

Then, in the following sections, a top down approach will be adopted for the description of the architectural design of the system:
\begin{description}
	\item[\ref{higharch}] \textbf{High-Level components}: A description of high-level components and their interactions.
	\item[\ref{componentview}] \textbf{Component View}: A detailed insight of the components described in the previous section.
	\item[\ref{deploymentview}] \textbf{Deployment view}: A set of indications on how to deploy the illustrated components on physical tiers.
	\item[\ref{runtimeview}] \textbf{Runtime View}: A thorough description of the dynamic behavior of the software with diagrams for the key-functionalities.
	\item[\ref{componentinterfaces}] \textbf{Component Interfaces}: A description of the different types of interfaces among the various described components.
	\item[\ref{archstyles}] \textbf{Selected Architectural styles and patterns}: A list of the architectural styles, design patterns and paradigms adopted in the design phase.
	\item[\ref{otherdecisions}] \textbf{Other design decisions}: A list of all other relevant design decisions that were not mentioned before.
\end{description}
