In this section each component of the high level components diagram is analysed in more details, explaining
its internal structure.

\defaultFigure[1.15]{LowLevelArchitecture.png}{Low level components diagram}
\begin{itemize}
    \item \textbf{Webserver} - 
    The webserver is containerized in a Docker Apache container and orchestrated by the \hyperlink{kubernetes}{Kubernetes service} for IBM.
    This composition allows to scale the server horizontally by just adding or removing containers with few click based on the load on the server.

    \item \hyperlink{cloudObjectStorage}{\textbf{Cloud Object Storage}}
    All the images will be stored on this component and will be accessible by a public URL. Metadata received from the computer vision service will be store here as metadata of the image they belong to.

    \item \textbf{Violations and Tickets}
    Receives violations from users and gives access to violations to officers and other microservices. It consists of a proprietary web application that exposes a REST API and is connected to a managed \hyperlink{postgres}{PostgreSQL database}.

    \item \textbf{Tickets hashing system}
    Stores the hash of the violations with automatic ticket option on a database that allows only inserts and reads.
    It provides an API for the \textit{Violations and tickets} service to POST a ticket, of which it will calculate the hash,
    and for the police to send and hash for a ticket that will be compared with the one stored in its database.

    \item \textbf{Statistics}
    A simple Python application that every day, when the load on the system is low, contacts the \textit{Violations and tickets} service to grab the violations for that day and builds new statistics that stores inside its \hyperlink{postgres}{Postgres database}.
    It also provides and API for users and officers to query for such statistics.

    \item \textbf{Suggestions}
    Uses the power of the \hyperlink{watson}{computer vision} and \hyperlink{discovery}{data mining} services to identify unsafe areas and to provide
    suggestions on how to fix them. It consists of a Python application that queries data from such services and provides an API for the local police to retrieve the
    suggestions found, which are stored in a relational database.

    \item \textbf{Other services}
    The \hyperlink{watson}{computer vision}, \hyperlink{discovery}{data mining} and \hyperlink{appid}{User management} services are well described in other sections and do not have any components relevant here since they are pay-per-use services.
\end{itemize}