All of our custom microservices will be stored in the Cloud offered by IBM and every external service is picked from the IBM Cloud Platform.
\begin{itemize}
	\item \textbf{Webserver}: The webserver is containerized in a Docker Apache container and orchestrated by the \hyperlink{kubernetes}{Kubernetes service} by IBM.
	This composition allows to scale the server horizontally just by adding or removing containers based on the load on the server.
	
	\item \hyperlink{cloudObjectStorage}{\textbf{Cloud Object Storage}}: All the SafeStreets' images will be stored on this component and will be accessible through a public URL. In addition, it allows us to receive metadata from the Computer Vision service that will be stored along with the image they belong to.
	
	\item \textbf{Violations and Tickets}: Receives violations from users and gives access to violations to officers and other microservices. This is our core service that consists in a proprietary web application exposing a REST API and connected to a managed \hyperlink{postgres}{PostgreSQL database}.
	
	\item \textbf{Tickets Hashing system}: Stores the hashes of the violations with automatic ticket option enabled on a database that allows only inserts and reads.
	It provides an API for the \textit{Violations and tickets} service to POST a ticket, of which it will calculate the hash, and for the police to send and hash for a ticket that will be compared with the one stored in its database.
	
	\item \textbf{Statistics}: This is a simple Python service that every day, when the load on the system is low, contacts the \textit{Violations and tickets} service to grab the violations for that day and builds new statistics that stores inside its \hyperlink{cloudant}{Cloudant database}.
	It also provides an API for users and officers to query such statistics.
	
	\item \textbf{Suggestion mining}: It uses the power of the IBM services related to \hyperlink{watson}{computer vision} and \hyperlink{discovery}{data mining} to identify unsafe areas and to provide suggestions on how to fix them. It consists of a Python application that queries and handle data from such services and provides an API for the local police to retrieve the suggestions found, which are stored in its database.
	
	\item \textbf{Tickets Checking}: Uses the \hyperlink{discovery}{data mining service} to identify potentially incorrect tickets and, if it is sure enough, it requires an officer to manually approve it.
	
	\item \textbf{Other services}
	The \hyperlink{watson}{computer vision}, \hyperlink{discovery}{data mining} and \hyperlink{appid}{User management} services are well described in other sections and do not have any components relevant here since they are pay-per-use services.
\end{itemize}