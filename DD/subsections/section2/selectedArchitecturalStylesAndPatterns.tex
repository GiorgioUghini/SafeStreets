\subsubsection{Microservices}
In a microservices architecture the system is divided into several autonomous services, each one is self-contained and implements a single business capability.

All the functionalities provided by the microservices to the final users are exposed through a REST API, while the internal communications service is provided, in our case, by IBM itself.

Some of the main advantages offered by this architecture are:
\begin{itemize}
    \item \textbf{Decoupling} - This allows to easily build, scale and alter services indepentently, which is particularly important for SafeStreets since it is difficult to foresee how it will scale.
    \item \textbf{Autonomy} - Developers can work indepentently on different services without complicated merges or the need to communicate too much, thus increasing speed.
    \item \textbf{Responsibility} - A single microservice does not focus on the entire application but on a single component that is handled as a product for which it is responsible
    \item \textbf{Decentralized Governance} - Each service can be developed with its own application stack that is the best for it, so you are not stuck with a single technology just because you started the project with it.
    \item \textbf{Agility} - Microservices support agile development. Any new feature can be quickly developed and discarded again.
\end{itemize}

\defaultFigure{microservices.jpg}{General scheme for a microservices architecture}

\subsubsection{Client - Server}
The most common pattern used on the web, it consists of several clients (mainly smartphones and computers) that contact a server which is always available to answer to requests and is reachable through its public IP.

In our case the concept of server is abstract because the application is not provided by a single, monolithic server but most principles remain.

\subsubsection{REST API}
REST stands for Representational State Transfer, is an architectural style for designing network application.
It is the most obvious choice to use with microservices if scalability is required because it forces statelessness.

REST relies on a client-server architecture and uses the HTTP protocol.