The logical division of the application consists of 3 layers that will be shown here: presentation, application and data.
\\Here we provide for each tier the definition, choice reasons and used technology:
\subsubsection{Presentation layer}\label{presentationlayer}
the presentation layer consists of a mobile application for the users and a website for the officers.
The application will be available from the major app stores while the website will be distributed through a web server.
It is important to note that both the application and the website will only provide the user interface and will retrieve data from the application layer.

\subsubsection{Application layer}\label{applicationlayer}
This layer will employ a microservices architecture.
Each microservice will handle a single functionality of the application in an atomic and stateless manner.
Also, each service will expose a REST interface accessible over HTTPS to be able to handle requests from the clients, the police systems and other microservices.

Microservices will be deployed in containers that will be able to efficiently scale based on the load on the single service, thus ensuring maximum scalability and elasticity and never wasting resources.
Since microservices are stateless by definition, redundancy can be easily implemented. This is a key point toward the availability requirement.

This architecture opens the possibility for some services to be bought instead of being implemented from scratch. For example the login/registration service will use the \hyperlink{appid}{App ID} service from IBM instead of a homemade solution.

The main services are:
\begin{itemize}
	\item \textbf{Users Management} - Login and registration
	\item \textbf{Violations and Tickets management}
	\item \textbf{Tickets Hashing system} - Allows the police to check for the integrity of a ticket by comparing an hash
	\item \textbf{Statistics}
	\item \textbf{Tickets checking} - Exploits computer vision and data mining to automatically check tickets, using information from different sources. Tickets that are suspected to be invalid are then sent to the local police for a double check.
	\item \textbf{Unsafe positions} - Uses an AI to extract unsafe positions from violations and accidents and to provide a possible solution to it. Such solutions are then available for the local police.
	\item \textbf{Computer Vision} - Extract info from pictures, like the car color and model
	\item \textbf{Data Mining} - Extracts patterns from huge amounts of data useful for other services
\end{itemize}

\subsubsection{Data layer}\label{datalayer}
Each microservice will have its own storage engine that cannot be directly contacted by other services. Pay-per-use services, such as \textit{Users management}, have their internal storage system that is abstracted,
for the services that need a storage system it will be hosted in the cloud and managed by IBM, billed per GB used. This system is the best possible abstraction for scalability.

All storage systems are guaranteed to have replicas and scheduled backups in order to avoid data loss.