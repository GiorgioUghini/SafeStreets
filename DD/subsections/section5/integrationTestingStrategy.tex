The first integration testing strategy that we'll use to test our application during its development, is the \textbf{Thread approach}. In particular, the \textbf{Bottom-up strategy} will be used to test and integrate modules within every thread. At first, only single portions of the modules (the ones that don't need any stubs) will be integrated and tested.
\\Small drivers needs to be created in order to give inputs to the portion of each module until a single feature is completed, then others threads will replicate the same procedure with the goal of reaching the completeness of the whole application.
\\Using a global strategy like this one will allow to have an application that works in the early stages of the implementation; this could permit to anticipate some testing and could minimize the costs of repair in the eventuality of an error.
\\Having said that, we will perform a second integration test when each module is fully implemented
\\In this new tests, we will impersonate a user who is actually using our application and who tries to get a result in response to a certain action.
\\This kind of test verifies not only the correct behavior of every single element, but also ensure the correct relationships with the other components of the application, listed in the above subsection.
\\The integration test that we will set up will use a real browser or application in order to perform, on the pages of the application, a certain number of actions in a programmatic way (such as the submission of a violation), to then verify a specific output at the end of the test. Following the example of the violation submission, the user will expect a successful message.
\\Although the integration test is undoubtedly the most complete and reliable, since it runs the entire stack during its execution, it is also the slowest, especially for the functionality that to be tested requires interaction with Computer Vision or Data Mining.
\\In our case, we would have needed to collect thousand of test results, but integrating the component with the thread approach while developing allow ou to proceed with fewer integration tests and then lower the execution times.

