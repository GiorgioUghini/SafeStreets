Here we explain how a possible good order of implementation of sub-components might help us to fulfill and build the application. Its achievement implies the following steps:
\begin{itemize}
	\item The very first thing to be implemented is the SafeStreets link with the municipality, implemented in form of encrypted HTTP APIs. The correct and secure definition of this link is something that the entire SafeStreets architecture relies on.
	\item At the second stage, we plan to create and configure all the server components/microservices (since they are those which will be queried by the client application afterwards).
	\\Every microservice component, will be implemented following the goals and functionalities described in this documents and in the SafeStreets RASD. The order of the implementation of each microservice is not relevant, but only the development of one microservice at a time will be allowed.
	Basically, our implementation order ranges from creating a user pool (IBM App ID) for data synchronization and authentication to setting up the cloud storage (IBM Cloud Object Storage and IBM Cloudant) for archiving data. 
	\item Then, when all microservice are implemented, we will focus on the creation of the client application such as the mobile app and the dashboard for the municipality.
\end{itemize}

As said, in our architecture, every microservice needs to be developed one small functionality at a time, and should be as atomic as possible. Once a microservice is ready, it will be tested on its own, in order to have a more efficient way of testing the application.
